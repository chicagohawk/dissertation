\documentclass[a4paper,onecolumn]{article}
\usepackage{amsmath, amsthm, graphicx, amssymb, wrapfig, fullpage, subfigure, array}
\usepackage[font=sl, labelfont={sf}, margin=1cm]{caption}
\DeclareMathOperator{\e}{e}
\linespread{2}

\begin{document}

\setcounter{page}{1}

\title{Significance and novelty of this paper}
\date{}
\maketitle
This paper proposes a method to estimate the gradient of objective functions 
constrained by conservation law simulations.
In previous works, adjoint method is widely used to compute the gradient.
But the adjoint method is not able to be implemented if the conservation law is unknown.
The present work enables adjoint computation for unknown conservation law by
leveraging the simulation's space- and/or time-solution. 
The method uses the solution to infer the conservation law, and then applies the adjoint method
to compute the gradient.
Furthermore, we demonstrate that the proposed method gives accurate
gradient estimation in two conservation law simulations.






\end{document}

