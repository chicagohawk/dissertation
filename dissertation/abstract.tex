% $Log: abstract.tex,v $
% Revision 1.1  93/05/14  14:56:25  starflt
% Initial revision
% 
% Revision 1.1  90/05/04  10:41:01  lwvanels
% Initial revision
% 
%
%% The text of your abstract and nothing else (other than comments) goes here.
%% It will be single-spaced and the rest of the text that is supposed to go on
%% the abstract page will be generated by the abstractpage environment.  This
%% file should be \input (not \include 'd) from cover.tex.


Many design applications can be formulated as optimization constrained by conservation laws. Such optimization can be efficiently solved by the adjoint method, which computes the gradient of the objective to the design variables. Traditionally, the adjoint method has not been able to be implemented in "gray-box" conservation law simulations. In gray-box simulations, the analytical and numerical form of the conservation law is unknown, but the full solution of relevant flow quantities is available. Optimization constrained by gray-box simulations can be challenging for high-dimensional design because the adjoint method is not directly applicable.\\


My thesis considers the gray-box models whose flux functions are unknown or partially unknown. 
I develop a twin model method that estimates the adjoint gradient from the gray-box space-time
solution. My method takes advantage of the big data, the gray-box space-time solution, to
infer the unknown part of the flux.
The solution is used to train a parameterized, adjoint-enabled conservation law simulator such that
a metric of solution mismatch is minimized.
After the training,
the twin model can estimate the gradient of the objective function by the adjoint method,
at a cost independent of the dimensionality of the gradient. 
In addition, an adaptive basis construction
procedure is presented for the training in order to fully exploit the information contained in the 
gray-box solution.
The availability of the estimated gradient enables more efficient optimization. My thesis 
considers a Bayesian optimization framework, in which 
the objective, the true gradient, and the error in the estimated gradient are 
modeled by Gaussian processes. 
Building upon previous research, a twin-model-enhanced Bayesian optimization algorithm 
is developed. I show that
the algorithm is able to find the optimum of the objective function
regardless of the gradient accuracy, if the 
true hyperparameters of the Gaussian models are known.\\

The twin model method and the twin-model-enhanced optimization are 
demonstrated in several gray-box models:
a Buckley-Leverett equation whose flux function
is unknown,
a steady-state Navier-Stokes equation whose state equation is unknown, 
and a porous media flow equation governing a petroleum reservoir whose componentwise
mobility factors are unknown. In these examples, the twin model is shown to
estimate the gradients accurately. 
Besides, the twin-model-enhanced Bayesian optimization is able to
achieve near-optimality within less iterations than without using the twin model.
Finally, I explore the applicability of the twin model method in
an example with one-thousand dimensional control by using a gradient descent approach.
The last example implies that the twin model may be adopted by
other optimization frameworks to improve convergence, which indicates a direction of future research.
