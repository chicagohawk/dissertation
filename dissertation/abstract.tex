% $Log: abstract.tex,v $
% Revision 1.1  93/05/14  14:56:25  starflt
% Initial revision
% 
% Revision 1.1  90/05/04  10:41:01  lwvanels
% Initial revision
% 
%
%% The text of your abstract and nothing else (other than comments) goes here.
%% It will be single-spaced and the rest of the text that is supposed to go on
%% the abstract page will be generated by the abstractpage environment.  This
%% file should be \input (not \include 'd) from cover.tex.


Many design applications can be formulated as optimization constrained by conservation laws. Such optimization can be efficiently solved by the adjoint method, which computes the gradient of the objective to the design variables. Traditionally, the adjoint method has not been able to be implemented in "gray-box" conservation law simulations. In gray-box simulations, the analytical and numerical form of the conservation law is unknown, but the full solution of relevant flow quantities is available. Optimization constrained by gray-box simulations can be challenging for high-dimensional design because the adjoint method is not directly applicable.\\

We consider the case where the flux function is unknown in the gray-box conservation law. The twin model method is presented to estimated the gradient by inferring the flux function from the space-time solution. The method enables the estimation of the gradient by solving the adjoint equation associated with the inferred conservation law. Building upon previous research, a Bayesian optimization framework is presented that admits the estimated gradient. The effectiveness of the proposed optimization method is compared to a conventional Bayesian optimization method where the gradient is unavailable. 
%The performance of the conventional method is found to deteriorate as the optimization dimensionality increases. 
The twin model enhances the Bayesian optimization performance given a limited number of gray-box simulations.

