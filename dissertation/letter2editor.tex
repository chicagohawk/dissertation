\documentclass[a4paper,twoside]{article}
\usepackage{amsmath, amsthm, graphicx, amssymb, wrapfig, fullpage, subfigure, array}
\usepackage[font=sl, labelfont={sf}, margin=1cm]{caption}
\usepackage{fancyhdr}
\pagestyle{fancy}
\fancyhead{}
\renewcommand{\headrulewidth}{1pt} % no line in header area
\fancyhead{} % clear all footer fields
\fancyhead[LE,RO]{\thepage}           % page number in "outer" position of footer line
\fancyhead[RE,LO]{\textbf{\textit{Author's response to reviewers,  JCOMP-D-15-01065}}} % other info in "inner" position of footer line
\usepackage[utf8]{inputenc}
\usepackage[english]{babel}
\usepackage[usenames, dvipsnames]{color}
\usepackage{ragged2e}
\hyphenpenalty=1000

\DeclareMathOperator{\e}{e}
\begin{document}
\setcounter{page}{1}

\textcolor{white}{1}

\vspace{1cm}
\centering
\textbf{Author's response to reviewers} \\
\vspace{1cm}
Journal of Computational Physics\\
JCOMP-D-15-01065\\
\vspace{.5cm}
Adjoint-based Gradient Estimation
Using the Space-time Solutions\\
of Unknown Conservation Law Simulations\\
\vspace{.5cm}
Han Chen, Qiqi Wang\\
\vspace{.5cm}
Department of Aeronautics and Astronautics, Massachusetts Institute of Technology,\\ Cambridge, MA, 02139, USA\\
\vspace{.5cm}
Corresponding author's email: voila@mit.edu\\

\justify
\noindent\makebox[\linewidth]{\rule{.75\paperwidth}{0.4pt}}\\

We are truly grateful to the reviewer for their time, effort, and insightful comments. 
We thank the associate editor Prof. Nicholas Zabaras for his effort in reviewing this manuscript.
We believe,
we have addressed all the issues raised by the reviewers in the revised paper. The following are the detailed
and itemized response to the reviewer's comments along with the information on where the corresponding changes have
been made in the revised paper.\\

\noindent\makebox[\linewidth]{\rule{.75\paperwidth}{0.4pt}}\\

\emph{\textbf{Comment 1:
The main point of problem considered in this paper is to estimate the unknown flux. The flux function will be modelled as a linear combination of some basis functions, which means that the choice of the basis functions should be based on the properties of the flux functions. So I am wondering that it is possible to summarize some general properties of the flux functions and give some suggestions on the choice of basis functions accordingly. In the 1-D example, the flux function is monotonically increasing for the specified range of the velocity, but not monotonically increasing in general.}}\\

There are many types of basis functions to parameterize a function, such as polynomial basis,
Fourier basis, and wavelet basis.
In this paper, we explain that a good choice for the basis is the sigmoid function.
There are two reasons for such a choice.\\

Firstly, we have Theorem 1 that studies on which domain of $u$ 
can the twin-model flux's derivative $\frac{\partial \tilde{F}}{\partial u}$ match the gray-box
flux's derivative, $\frac{\partial {F}}{\partial u}$, for a one-equation PDEs with one spatial
dimension. We prove that if the two solutions match, 
then it is guaranteed that  $\frac{\partial \tilde{F}}{\partial u}$  and
 $\frac{\partial {F}}{\partial u}$ match in a finite-supported domain
$B_u$ where the gray-box solution exists and has large enough variation. 
For PDEs having more-than-one equation and/or more-than-one spatial dimension,
we expect (without proof) such a domain also has a finite support.
In addition, if the solution is discretized, we hold the belief (without proof) that they may 
match better on a domain where the gray-box discretized solution is more densely sampled.
Therefore, a suitable basis function should allow local refinements on a certain domain. 
``Global'' basis, such as polynomial and Fourier,
does not satisfy this requirement.
A type of basis that allows local refinment is the wavelet basis.\\

Secondly, if the twin model flux $\tilde{F}$ and the gray-box model flux $F$
are off by a constant, i.e. $\tilde{F} = F+a$, then $\frac{\partial \tilde{F}}{\partial u}=
\frac{\partial {F}}{\partial u}$. Thus the solution of the twin model and the gray-box
model to any initial value problem will be the same. In addition, the solutions to any adjoint
equation, with any objective function, will be the same. As a result, the gradient estimated
by the twin model will be the true gradient. Therefore, $\frac{\partial \tilde{F}}{\partial u}$,
instead of $\tilde{F}$, should be approximated. So the basis for univariate $\tilde{F}$ shall be
chosen as the indefinite integral of the regular basis functions. 
The indefinite integral of wavelet is sigmoid function.\\

Based on these two reasons, we recommend sigmoid functions as the bases of univariate twin-model 
fluxes. For multivariate fluxes, the tensor product of sigmoid functions are recommended in this
paper.\\

In the older version of our paper, we explained that the monotonicity of a flux can be enforced
by setting the coefficients of all sigmoid bases to be positive. The monotonicity is 
not generally applicable, so we removed such discussion in our revised version.\\

\emph{\textbf{Comment 2:
To approximate the flux function, a range of the value of the physical quantities needs to be specified. It is not mentioned in the paper how such a range should be determined. I think it is an importance issue. First, the flux function will be defined on this range, which means that this range is related to the number of coefficients for a good approximation. Second, if the range is 1 2 too small, the twin model may fail when it generates a value that is out of the range.}}\\

The comment involves two issues. The first issue is how to determine the range of solution
on which $\tilde{F}$ will be defined. A range that is too wide may result in unnecessary 
coefficients and possibly introduce overfitting. 
A range that is too narrow may undermine the flux estimation
accuracy and possibly introduce underfitting.
The second issue is how many bases shall be used, where to specify the bases
(corresponds to $\frac{\eta}{2^j}$ in the sigmoid basis),
and how to specify the resolution of the bases (corresponds to $j$ in the sigmoid basis).
The two issues are closely related because they both reflect the fact that
the bases shall be determined adaptively according to the gray-box solution.\\

In the revised version of our paper, we address the two issues all together by introducing
an adaptive basis construction scheme.\\

Bases can be determined adaptively through methods like Lasso regularization 
\cite{Lasso variable selection},
matching pursuit \cite{match pursuit}, and basis pursuit \cite{L1 basis pursuit}. 
In the revised paper, we review such methods in Section 
3.2. Such methods select a subset of ``significant'' bases from a basis dictionary (the 
definition of basis dictionary is in Section 3.2). 
Conventionally, the dictionary needs to be fixed and pre-determined, with the belief that it is a 
superset of the significant bases.
This can be problematic we are unable to ensure, a prior, that a fixed and pre-determined dictionary 
contains all the significant bases for approximating the unknown flux function.\\

To address this issue, methods have been devised that construct the dictionary adaptively. 
Our revised paper adopts the approach proposed in \cite{adaptive basis 1} and \cite{adaptive basis 2}.
Starting from some trivial bases, 
a dictionary is built up progressively by iterating over a forward step and a backward step.
The forward step searches over a candidate set of bases, and adds the significant ones 
to the dictionary.
The backward step searches over the current dictionary, and removes the 
insignificant ones from the dictionary.
The iteration stops only when no alternation is made to the dictionary or when some criterion
achieves a target value.\\

To enable the forward-backward procedure, three elements are developed in Section 4.3:
1) A formulation is provided to efficiently assess the significance of each candidate basis.
The assessment requires only a numerical quadrature, without extra conservation law simulations.
2) A concept of the a sigmoid basis' ``neighborhood'' is developed. 
The neighborhood defines the the candidate bases at each forward step.
3) A criterion based on cross validation is devised that determines when to add or remove a 
candidate basis in the forward or backward step. The complete algorithm is summarized in
Section 4.4.\\

Section 5.1 demonstrates the adaptability of the selected bases in a numerical example. The 
adaptive scheme enables more accurate gradient estimation compared with an ad hoc set of bases.
Therefore, we believe the two issues raised by the reviewer have been addressed after the revision.\\

\emph{\textbf{Comment 3:
The mismatch M is nothing but the distance the real solution and the model solution, which should be defined with respect to the norm used for the spatial discretization of the real solution.}}\\

In the older version of our paper, the solution mismatch is defined as
$$
    \mathcal{M} = \frac{1}{T} \sum_{i=1}^N \sum_{j=1}^M \left(\tilde{\boldsymbol{u}}_{ij}
    - \boldsymbol{u}_{ij}\right)^2 \Delta t_j |\Delta \boldsymbol{x}_i|\,.
$$
The notations are described in the beginning of the revised paper. 
In the revised version, the solution mismatch is defined more compactly and more generally as
\begin{equation*}
    \mathcal{M} = \sum_{i=1}^M \sum_{j=1}^N w_{ij} \big( \tilde{\boldsymbol{u}}_{ij}
     -\boldsymbol{u}_{ij}\big)^2 \,,
    \label{eqn: solution mismatch}
\end{equation*}
where $w_{ij}$'s are the space-time quadrature weights. Unfortunately, we had a typo
in defining the solution mismatch in the return bend numerical example:
we defined mistakenly the solution mismatch to be
$$
    \mathcal{M} = w_\rho \|\tilde{\rho}_\infty - \rho_\infty\|^2 
    + w_u \|\tilde{u}_\infty - u_\infty\|^2 
    + w_v \|\tilde{v}_\infty - v_\infty\|^2
    +w_E \|\tilde{E}_\infty - E_\infty\|^2\,.
$$
Indeed, $\|\cdot\|$ should be $\|\cdot\|_Q$, the norm with respect to the quadrature associated with 
the spatial discretization, which is in line with our definition of $\mathcal{M}$.
In the revised paper we correct the mistake. We apologize for the typo in the older version.\\


\emph{\textbf{Comment 4:
The authors proposed an intuitive choice for the weight constants in the M given in equation (29). Mathematically speaking, as long as the weights are positive number, they are equivalent. I am not sure the advantage of the intuitive choices.}}\\

There can be more-than-one equations in a conservation law. For example, in
the return bend numerical example, we have equations for the density, the
x- and y-directional momentums (or velocities), 
and the energy density. The solution mismatch, $\mathcal{M}$, is a combination of the mismatches
of all these quantities. 
Therefore, a positive weight shall be assigned to each of the mismatches in defining $\mathcal{M}$.\\

Different choices of the weight results in different $\tilde{F}$ and different gradient estimation.
To see this, we view $\mathcal{M}$ as a function of $\tilde{F}$ and the weight $w$, i.e.
$\mathcal{M} = \mathcal{M}(\tilde{F}, w)$. Given a $w$, the trained twin model approximately satisfies
$\left.\frac{\partial \mathcal{M}}{\partial \tilde{F}}\right|_w = 0$. Therefore,
we obtain by variation that
$$
    \delta \tilde{F} = - \left(\frac{\partial^2 \mathcal{M}}{\partial \tilde{F}^2}\right)^{-1}
    \left(\frac{\partial^2 \mathcal{M}}{\partial \tilde{F}\partial w}\right) \delta w\,.
$$
Besides, the estimated gradient $G\equiv \frac{\partial \tilde{\xi}}{\partial c}$ 
is a function of $\tilde{F}$.
Denote the variation of the estimated gradient by
$$
    \delta G = \frac{dG}{d\tilde{F}} \delta \tilde{F}\,.
$$
Thereby,
\begin{equation*}
    \delta G = - \frac{dG}{d\tilde{F}} \left( \frac{\partial^2 \mathcal{M}}{\partial \tilde{F}^2} \right)^{-1} \left( \frac{\partial^2 \mathcal{M}}{\partial \tilde{F} \partial w} \right) \delta w 
    \equiv - \Lambda \delta w
\end{equation*}
The estimated gradient indeed depends on the choice of $w$.
In theory, this linear relationship enables one to choose an optimal $w$ depending on her goal. 
For example, if the goal is to minimize the sensitivity of $G$ with respect to $w$, 
we may choose a $w$ that minimizes $\|\Lambda^T \Lambda\|$, where $\|\cdot\|$ indicates the
induced norm. In practice, however, it can be costly to find such an optimal weight due to 
the Hessian in the formulation of $\Lambda$.\\

In the older version,
we first set $\tilde{F}$ as some randomly guessed functions, then obtained an ensemble of solution
mismatches for different physical quantities. The weights were set to be the 
inverse of the ensemble-averaged solution mismatches.
For example, in the return bend numerical example, the weights for the density, velocities, and
the energy density were chosen as
\begin{equation*}
    w_\rho = \frac{1}{\Big< \|\tilde{\boldsymbol{\rho}}_\infty - \boldsymbol{\rho}_\infty\|_Q^2 \Big>}\,,
    \; 
    w_u = \frac{1}{\Big< \|\tilde{\boldsymbol{u}}_\infty - \boldsymbol{u}_\infty\|_Q^2 \Big>}\,,
    \;
    w_v = \frac{1}{\Big< \|\tilde{\boldsymbol{v}}_\infty - \boldsymbol{v}_\infty\|_Q^2 \Big>}\,,
    \;
    w_E = \frac{1}{\Big< \|\tilde{\boldsymbol{E}}_\infty - \boldsymbol{E}_\infty\|_Q^2 \Big>}\,.
\end{equation*}
In the revised version, we set the weights as the normalization constants.
It is observed that our older treatment slightly increases the converging 
speed in training the twin model,
although the trained twin model gives similar estimated gradient. We are unable to justify the
advantage of each choice. Therefore we specify the weights simply as the normalization constants
to avoid possible confusions. It remains a future work to study the choices and the effects
of the weights.\\


\emph{\textbf{Additional note:}}
We will also submit another paper, 
\emph{Twin-model Enhanced Bayesian Optimization Constrained by Gray-box Conservation Law Simulations},
as a companion paper.
The companion paper will utilize
the estimated gradient, obtained through the twin model developed in this paper, 
to enhance the Bayesian optimization performance. 



\begin{thebibliography}{9}

\bibitem{Lasso variable selection}
Tibshirani, R.
\newblock Regression Shrinkage and Selection via the Lasso.
\newblock {\em Journal of the Royal Statistical Society}, Series B (Methodological):267-288, 1996.

\bibitem{match pursuit}
Mallat, S. G., and Zhang, Z.
\newblock{ Matching Pursuits with Time-Frequency Dictionaries.}
\newblock{\em Signal Processing, IEEE Transactions on}, 41(12):3397-3415, 1993.

\bibitem{L1 basis pursuit}
Chen, S. S., Donoho, D. L., and Saunders, M. A.
\newblock{Atomic Decomposition by Basis Pursuit.}
\newblock{\em SIAM Review}, 43(1):129-159, 2001.

\bibitem{adaptive basis 1}
Jekabsons, G.
\newblock Adaptive Basis Function Construction: an Approach for Adaptive Building of Sparse Polynomial Regression Models.
\newblock {\em INTECH Open Access Publisher}, 2010.

\bibitem{adaptive basis 2}
Blatman, G., and Sudret, B.
\newblock Sparse Polynomial Chaos Expansions and Adaptive Stochastic Finite Elements Using a Regression Approach.
\newblock {\em Comptes Rendus M$\check{\textrm{e}}$canique}, 336(6):518-523, 2008.

\bibitem{cross validation}
Geisser, S.
\newblock {Predictive inference},
\newblock {\em CRC press}, vol. 55, 1993.

\end{thebibliography}


\end{document}

