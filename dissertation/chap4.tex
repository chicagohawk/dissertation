%% This is an example first chapter.  You should put chapter/appendix that you
%% write into a separate file, and add a line \include{yourfilename} to
%% main.tex, where `yourfilename.tex' is the name of the chapter/appendix file.
%% You can process specific files by typing their names in at the 
%% \files=
%% prompt when you run the file main.tex through LaTeX.
\chapter{Conclusions}
\label{chapter 4}
In this thesis, I addressed the optimization constrained by gray-box simulations.
I enabled the adjoint gradient computation for gray-box simulations by leveraging the
space-time solution. In addition, I utilized the gradient information in a Bayesian
framework to faciliate a more efficient optimization. 
To conclude, this chapter summarizes the developments and highlights
the contributions of this work. I close with suggestions for continuing work
on this topic.


\section{Thesis Summary}
Optimization constrained by conservation law simulations are prevalent in many engineering
applications.
In many cases, the code of the simulator is proprietary, legacy, and
lacks the adjoint capability. 
Chapter \ref{chap 1} categorizes such simulators as gray-box.
The gray-box scenario limits the efficient application 
of gradient-based optimization methods.
I motivate the need for the
adjoint gradient, and explain the feasibility
of estimating the adjoint gradient in the gray-box scenario.
The key is to leverage the gray-box space-time solution, which contains information
of the gray-box simulator but is usually abandoned by conventional optimization methods.
To restrict the scope of my thesis, a class of problems is formulated where the
flux functions are partially unknown.\\

To address this issue, an adjoint-enabled twin model is proposed 
to match the space-time solution.
In Chapter \ref{chapter 2}, I develop a two-stage procedure to estimate the gradient.
In the first stage, a twin model is trained to minimize the solution mismatch.
In the second stage, the trained twin model
computes an adjoint gradient which approximates the true gray-box gradient.
For a simple conservation law with only one equation and one dimensional space,
I demonstrate theoretically that the twin model can indeed infer the gray-box 
conservation law on a domain that has large solution variation.
To implement the twin model numerically,
the unknown part of the flux function is parameterized by a set of bases.
I argue that the sigmoid bases are well suited for this problem since
their gradients are local. The procedure is demonstrated
on a Buckley-Leverett equation using a set of sigmoids chosen manually.
Although the estimated gradient is accurate, several limitations are observed
which lead to the developments of adaptive basis construction.
Several tools are introduced for the adaptive basis construction,
including a metric of the basis significance, the basis neighborhood, 
and the cross validation.
The adaptive basis construction 
fully exploits the information contained in the
gray-box solution, and avoids the problem of overfitting.
Based upon these developments, a twin model algorithm is presented.
The algorithm selects the basis dictionary adaptively using a forward-backward iteration
procedure, where either the solution mismatch or the integrated truncation error
can be used as the metric for basis selection.
The twin model algorithm is demonstrated on 
several numerical examples:
a 1D convection equation with unknown 
flux function, a 2D steady-state Navier-Stokes flow with unknown state equation,
and a 3D petroleum reservoir flow with unknown mobility factors.
In all the three examples, the twin model algorithm provides accurate estimation
of the true gradient, which represents a major contribution towards enabling the
adjoint gradient computation for gray-box simulations.\\

Using the twin-model gradient, optimization can be done more efficiently.
Chapter \ref{chapter 3} incorporates the twin-model gradient into a Bayesian optimization framework,
in which the objective function, the true gradient,
the estimated gradient, and the gradient error are modeled by Gaussian processes.
The model provide analytical expressions for the 
posterior distributions and the acquisition function, while the 
hyper parameters are estimated by maximum likelihood.
I present a Bayesian optimization algorithm that utilizes the twin-model gradient.
In addition, I show that the algorithm is able to find the optimal control regardless
of the gradient estimation accuracy, if the true hyper parameters are used.
The optimization algorithm is demonstrated on several similar problems discussed in
Chapter \ref{chapter 2}:
a Buckley-Leverett equation with source term controls,
a Navier-Stokes flow in a return bend with boundary geometry controls,
and a petroleum reservoir with polymer-water injection rate controls.
In all the three examples, the
twin-model optimization achieves
near-optimality with less iterations than the vanilla Bayesian optimization without
the gradient information,
which represents another major contribution of my thesis.
Finally, the twin model gradient is tested on a
high-dimensional control problem, by employing a simple gradient descent approach. 
The gradient efficiently enables the optimization of the high-dimensional problem.\\


\section{Contributions}
The main contributions of this work are:
\begin{enumerate}
    \item a twin model algorithm that enables the adjoint gradient computation for gray-box
          conservation law simulations;
    \item an adaptive basis construction scheme that fully exploits the information of
          gray-box solutions and avoids overfitting;
    \item a Gaussian process model of the twin-model gradient
          and a Bayesian optimization algorithm that employs the twin model; and
    \item theoretical and numerical demonstrations of the algorithms 
          in several numerical examples: the Buckley-Leverett equation, the
          Navier-Stokes equation, and the porous media flow equation.
\end{enumerate}

\section{Future Work}
There are several potential thrusts of further research: 
A useful extension is to investigate the
inferrability of twin models for various conservation laws. In particular,
Theorem \ref{theorem: 1} may be extended for more general problems with a
system of equations and higher spatial dimension. 
%Another important direction
%is to study the applicability of using the integrated truncation error
%for adaptively constructing the basis dictionary. especially to
%obtain a necessary and sufficient condition for bounding $\mathcal{M}$
%with $\mathcal{T}$.
Besides, in the twin-model Bayesian optimization algorithm, 
it is of great practical value to reuse twin model more efficiently.
My current approach uses the basis dictionary of the twin model in 
the last iteration as an initial guess
of the basis dictionary in the current iteration,
then re-trains the twin model. In the future, a
research direction is on how to utilize all previously trained twin models,
for example by employing
the ``trust region'' technique in the optimization: the same twin model can be used multiple times
at different controls inside a trust region of the control space\footnote{In my thesis,
the twin model is re-trained at each new control. Generally, gradient-based
trust region methods require the gradient to satisfy a property called full-linearity
\cite{trustregionconn, trustregionwild}.
Unfortunately, this property is not guaranteed 
by the twin-model gradient. The lack of full-linearity is a key factor
that refrains me from exploring the trust-region methods in my thesis. It's an open question
on how to introduce the trust region framework into the optimization}, thus reducing
the training cost.
%Finally, it is interesting to generalize the formulation \eqref{eqn: govern PDE} to
%incorporate unknown source terms and boundary conditions.\\

