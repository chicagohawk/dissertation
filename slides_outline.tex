\documentclass[a4paper,onecolumn]{article}
\usepackage{amsmath, amsthm, graphicx, amssymb, wrapfig, fullpage, subfigure, array,float}
\usepackage[]{algorithm2e}
\usepackage[toc, page]{appendix}
\usepackage{pdfpages, nomencl, pifont}
\usepackage[nottoc, numbib]{tocbibind}
\usepackage{tikz}
\usetikzlibrary{positioning,shadows,arrows}
\usepackage[font=sl, labelfont={sf}, margin=1cm]{caption}
\DeclareMathOperator{\e}{e}
\newtheorem{definition}{Definition}
\theoremstyle{remark}
\newtheorem{theorem}{Theorem}
\newtheorem{remarker}{Remark}
\makenomenclature

\begin{document}
\setcounter{page}{1}
\hspace{.36\textwidth}
\Large\textbf{Slides Outline}\\
\normalsize

\textbf{Sections}
\begin{enumerate}
    \item Research background.
    \item Twin model.
    \item Optimization with twin model.
    \item Application to a turbulent flow optimization.
\end{enumerate}

\newpage
\textbf{Research background}
\begin{enumerate}
    \item Motivation: 
          \begin{itemize}
              \item Optimization based on conservation law simulation. 
              \item Design space is high dimensional.
          \end{itemize}
          Example: optimization of turbulent return bend geometry. (picture)
    \item Many conservation law simulators are gray-box. What is gray-box / black-box / open-box. (table, merge PDE+implementation)
    \item Research scope:
          \begin{itemize}
              \item conservation law.
              \item gray-box simulation.
              \item high-dimension design.
          \end{itemize}
    \item If open-box, review of gradient-based.
    \item In reality, not open-box. Treat as black-box, review of gradient-free. Curse of dimensionality.
          However, not take advantage of space-time solution.
    \item When lower-fidelity models are available, can save overall computation time by multi-fidelity optimization.
          High-fidelity model is the gray-box simulator; low-fidelity model can be constructed by surrogate methods.
    \item Review of surrogate methods. 
          \begin{itemize}
              \item physics-based surrogates: use the physics of the underlying system, such as
                    \begin{itemize}
                        \item coarser discretization: \emph{require simulator's PDE.}
                        \item reduced order model (POD, DEIM, balanced truncation): \emph{require simulator's PDE.}
                        \item simplified physics (RANS, dual porosity, thin airfoil theory).
                    \end{itemize}
              \item functional surrogate: use the sample value of objective function, such as
                    \begin{itemize}
                        \item radial basis function approximation.
                        \item polynomial approximation.
                        \item neural network.
                    \end{itemize}
          \end{itemize}
          Narrative: functional surrogate not suitable in my research scope because design space is high-dimensional. Focus on physics-based surrogate.
    \item Physics-based surrogates can be adaptive.
          \begin{itemize}
              \item Adaptive discretization: \emph{require simulator's PDE.}
              \item Goal oriented reduced order model: match observations, \emph{require simulator's PDE}
          \end{itemize}
    \item Research objective.
          \begin{itemize}
              \item Develop a method for inferring a surrogate conservation law from the space-time solution
                    of an gray-box simulation.
              \item Assess how much computation saving can be achieved using the proposed method in a return bend optimization problem.
          \end{itemize}
\end{enumerate}

\newpage
\textbf{Twin model}
\begin{enumerate}
    \item Adaptive physics-based surrogate that reproduces the space-time solution.
          \begin{itemize}
              \item does not require simulator's PDE, instead infer a convervation law PDE.
              \item PDE is inferred by matching the gray-box simulation's space-time solution.
          \end{itemize}
          Example problem: infer Buckley-Leverett equation flux function.
    \item Benefits of matching space-time solution
          \begin{itemize}
              \item Small domain of dependence (low dimensional inputs)
              \item Large number of samples
              \item Dimension independent of design
          \end{itemize}
          Time independent is a special case. Pseudo-time marching until convergence.
    \item Functions can be parameterized. Infer twin model is an inverse problem.\\
          Review of inverse problems, two viewpoints.
          \begin{itemize}
              \item Statistical estimation.
              \item Parameter identification.
          \end{itemize}
          Twin model is an open-box, use adjoint for parameter identificaiton.
    \item Formulation. Time dependent. General: not same grid, different weight.
    \item Formulation. Time-independent. General.
    \item Candidate basis (Questions)
          \begin{itemize}
              \item polynomial
              \item Fourier
              \item wavelet
          \end{itemize}
          How about adaptive basis selection (supervised: peceptron basis selection. unsupervised: e.g. vector quantization.)
    \item Numerical example: twin model with fixed structure
    \item Numerical example: prediction of objective functions about one training.
    \item Basis selection: for time-dependent and time-independent twin models
          \begin{itemize}
              \item regularization-based methods
              \item test-based methods
          \end{itemize}
    \item Basis selection: for time-dependent twin models.
          Reduce to linear basis selection scheme by 
          \begin{itemize}
              \item Solve twin model with restart.
              \item Finite difference approximation.
          \end{itemize}
          (picture: basis selection)
\end{enumerate}

\newpage
\textbf{Optimization with twin model}
\begin{enumerate}
    \item General optimization statement.
    \item Review of multi-fidelity optimization schemes. Provably convergent.
          \begin{itemize}
              \item pattern search
              \item Trust region
              \item Bayesian
          \end{itemize}
    \item Review of Bayesian MFO.
          Include Bayesian optimization flowcharts.
    \item Advantages of Bayesian MFO
          \begin{itemize}
              \item Uses all high-fidelity evaluations to choose the next design to evaluate.
              \item Balances exploration and exploitation. Done by appropriate acquisition function of the posterior.
                    Expected improvement formulation.
          \end{itemize}
    \item Propose to use twin model's gradient and gray-box simulation objective function in
          Bayesian multifidelity framework.
          (flowchart)
    \item Model twin model gradient and gray-box objective function.
          Posterior can be obtained by coKriging. Hyperparameters can be estimated by MLE.
    \item Flowchart (full optimization procedure)
    \item Demonstration: high-dimensional optimization with noisy gradient.
    \item Discussion on convergence of non-constraint optimization. Proposed theorem.
    \item Constraint optimization. Two types of constraints (their names in existing literatures?)
          \begin{enumerate}
              \item Type 1: require solution of PDE simulation.
              \item Type 2: not require solution of PDE simulation.
          \end{enumerate}
          Only interested in type 1.
    \item Constraint optimization. Constraint expected improvement formulation.
\end{enumerate}

\newpage
\textbf{Application to a turbulent flow optimization}
\begin{enumerate}
    \item Problem description.
          Optimize return bend geometry to minimize pressure drop. Low Mach number, incompressible.
          (picture)
          Interested in ensemble average quantities. 
    \item Candidate PDE models for turbulent flows. Two categories:
          \begin{itemize}
              \item PDEs for ensemble average quantities. They satisfy a conservation law PDE.
              But the PDE is not closure: modelling Reynolds stress is an open problem. RANS models.
              Low-fidelity.
              \item PDEs for instantaneous quantities. Such as LES, DNS. Costly to solve.
              Generally no adjoint? High-fidelity.
          \end{itemize}
    \item Propose to construct a twin model for the ensemble average quantities. Train the twin model
          by high fidelity simulation's time-averaged space-time solution.
          (flowchart)
    \item Review of RANS models.
          \begin{itemize}
              \item Models based on Boussinesq hypothesis: eddy viscosity.
                    \begin{itemize}
                        \item algebraic models.
                        \item one-equation models. (transport equation)
                        \item two-equation models. (transport equation)
                    \end{itemize}
              \item Reynolds stress modelling.
          \end{itemize}
          Focus on two-equation models for its ability to capture more complex turbulent phenomenons.
    \item Twin model can be applied to infer PDE for eddy viscosity. For example $k-\omega$ models. The rhs of PDEs for
          for $k$ and $\omega$ are adaptive.
    \item Future work.
          \begin{itemize}
              \item Implement LES in OpenFoam, serving as gray-box simulation.
              \item Implement a RANS model with adjoint.
              \item Still an open problem how to parameterize the rhs of twin model.
                    As a first step, consider the constants in the $k-\omega$ model as variables.
              \item Assess computational savings, and optimal solution quality given limited computation budget.
          \end{itemize}
\end{enumerate}

\newpage
\textbf{Miscellaneous}
\begin{enumerate}
    \item Expected contributions
          \begin{itemize}
              \item Enable adjoint gradient computation even if the governing conservation law is not available.
              \item Demonstrate that efficient gradient-based optimization is possible, even if the underlying simulator does not
                    implement an adjoint.
              \item Demonstrate my method's superiority in a high-fidelity turbulent flow optimization with a high-dimensional design space.
          \end{itemize}
    \item Proposed schedule
          \begin{itemize}
           \item   \textbf{Completed:}
                   \begin{itemize}
                       \item Course work.
                       \item Formulation of twin model and its inference.
                       \item Basis selection for time-dependent twin model.
                       \item Optimization framework using twin model.
                       \item Demonstration of twin model in a 1D example.
                   \end{itemize}
           \item   \textbf{To be complete:}
                   \begin{itemize}
                       \item \textbf{2015 Apr}: Write proposal defense slides. Wrap up the proof of theorem \ref{theorem: 2}.
                       \item \textbf{2015 May}: Defend proposal. Demonstrate the complete algorithm on a the 1D Buckley-Leverett example.
                       \item \textbf{2015 Jun}: Setup an LES solver for the return bend testcase in OpenFoam, write a RANS solver with adjoint for the return bend.
                       \item \textbf{2015 Jul}: Hold a committee meeting for progress report.
                       \item \textbf{2015 Jul-Oct}: Implement optimization on the return bend example.
                       \item \textbf{2015 Aug-Nov}: Write thesis.
                       \item \textbf{2016 Jan}: Defend thesis.
                   \end{itemize}
          \end{itemize} 
    \item Major and minor programs of study, courseworks
    \item References
\end{enumerate}

\end{document}
